\titledquestion{Hessian coloring}

Consider the function
\[
	f : \mathbb{R}^{7} \to \mathbb{R} : f(x) = \sum_{i=1}^3 x_{2i}(x_{2i-1} - x_{7}) + \sum_{i=1}^{7} x_i^2
\]
Our aim is to compute the Hessian $\nabla^2 f(x)$ by exploiting its sparsity.

\begin{itemize}
	\item Find the adjacency graph of the Hessian matrix:
	
	\begin{solutionbox}{8.5cm}
		\begin{center}
			\includegraphics[height=8cm]{adjacency_graph.png}
		\end{center}
	\end{solutionbox}
	
	\item How many colors would be used with \textbf{star} coloring ?

	\begin{solutionbox}{8.5cm}
		\begin{center}
			\includegraphics[height=7cm]{star_coloring.png} \hspace{3cm} \framebox{3}
		\end{center}
	\end{solutionbox}

	\item
	How would the nonzero entries of the Hessian be obtained from
	Hessian-Vector Products using this star coloring ?
	
	\begin{solutionbox}{11cm}
%		The Hessian matrix is
%		\[
%		\begin{pmatrix}
%		a_{11} & a_{12} & 0      & 0      & 0      & 0      & 0      \\
%		a_{12} & a_{22} & 0      & 0      & 0      & 0      & a_{27} \\
%		0      & 0      & a_{33} & a_{34} & 0      & 0      & 0      \\
%		0      & 0      & a_{34} & a_{44} & 0      & 0      & a_{47} \\
%		0      & 0      & 0      & 0      & a_{55} & a_{56} & 0      \\
%		0      & 0      & 0      & 0      & a_{56} & a_{66} & a_{67} \\
%		0      & a_{27} & 0      & a_{47} & 0      & a_{67} & a_{77}
%		\end{pmatrix}
%		\]
		The three Hessian-Vector Products (HVPs) corresponding to the two colors are:
	
		\begin{align*}
			H\begin{bmatrix} 1\\0\\1\\0\\1\\0\\0 \end{bmatrix} &=
			\begin{bmatrix}
				a_{11}\\
				a_{12}\\
				a_{33}\\
				a_{34}\\
				a_{55}\\
				a_{56}\\
				0
			\end{bmatrix} &
			H\begin{bmatrix} 0\\1\\0\\1\\0\\1\\0 \end{bmatrix} &=
			\begin{bmatrix}
				a_{12}\\
				a_{22}\\
				a_{34}\\
				a_{44}\\
				a_{56}\\
				a_{66}\\
				a_{27} + a_{47} + a_{67}
			\end{bmatrix} &
			H\begin{bmatrix} 0\\0\\0\\0\\0\\0\\1 \end{bmatrix} &=
			\begin{bmatrix}
				0\\
				a_{27}\\
				0\\
				a_{47}\\
				0\\
				a_{67}\\
				a_{77}
			\end{bmatrix}
		\end{align*}


		The diagonal entry $a_{ii}$ is obtained from the HVP of the
		$\phi(i)$-th color.
		The off-diagonal entry $a_{ij}$ of a star with center $j$ in the subgraph of colors $\phi(i)$ and $\phi(j)$ is obtained
		from the $i$-th entry of the HVP of the $\phi(j)$-th color.

		In this case,
		the off-diagonal entries $a_{27}, a_{47}, a_{67}$
		are obtained from the rows $2, 4, 6$ respectively of
		the HVP $He_7$.
		For the off-diagonal entries $a_{12}, a_{34}, a_{56}$,
		the 2-color subgraph is a simple edge so both nodes can be treated as center of the star.
		So they can either be obtained from the rows $1, 3, 5$ respectively of
		the HVP $H(e_2+e_4+e_6)$ or from the rows $2, 4, 6$ respectively of
		the HVP $H(e_1+e_3+e_5)$.
	\end{solutionbox}

	\item How many colors would be used with \textbf{acyclic} coloring ?

	\begin{solutionbox}{8cm}
		\begin{center}
			\includegraphics[height=7cm]{acyclic_coloring.png} \hspace{3cm} \framebox{2}
		\end{center}
	\end{solutionbox}

	\item How would the nonzero entries of the Hessian be obtained from the
	Hessian-Vector Products using this acyclic coloring ?
	
	\begin{solutionbox}{10cm}		
		\begin{align*}
			H\begin{bmatrix} 0\\1\\0\\1\\0\\1\\0 \end{bmatrix} &=
			\begin{bmatrix}
				a_{12}\\
				a_{22}\\
				a_{34}\\
				a_{44}\\
				a_{56}\\
				a_{66}\\
				a_{27} + a_{47} + a_{67}
			\end{bmatrix} &
			H\begin{bmatrix} 1\\0\\1\\0\\1\\0\\1 \end{bmatrix} &=
			\begin{bmatrix}
				a_{11}\\
				a_{12} + a_{27}\\
				a_{33}\\
				a_{34} + a_{47}\\
				a_{55}\\
				a_{56} + a_{67}\\
				a_{77}
			\end{bmatrix}
		\end{align*}

		The diagonal entry $a_{ii}$ is obtained from the HVP of the
		$i$th color.
		The off-diagonal entry $a_{ij}$ of a leaf $i$ in the subgraph of colors $\phi(i)$ and $\phi(j)$ is obtained
		from the $i$-th entry of the HVP of the $\phi(j)$-th color.
		We then substract their values from the row $j$ of the HVP of color $\phi(i)$ and
		remove this edge from the subgraph to allow new leaves to emerge and continue the process.

		In this case,
		the off-diagonal entries $a_{12}, a_{34}, a_{56}$
		are obtained from the rows $1, 3, 5$ resp. of
		the HVP $H(e_2+e_4+e_6)$.
		Their values are then substracted from the
		rows $2, 4, 6$ resp. of
		the HVP $H(e_1+e_3+e_5+e_7)$.
		The off-diagonal entries $a_{27}, a_{47}, a_{67}$
		are then obtained from these respective rows.
	\end{solutionbox}
\end{itemize}
