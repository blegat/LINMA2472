\titledquestion{Implicit}

Consider the function
\[
	f(x, \lambda) : x^3 - \lambda x^2 + x - 1
\]
If $|\lambda| < \sqrt{3}$,
the derivative $\partial_1 f(x, \lambda) = 3x^2 - 2\lambda x + 1$
is positive for all $x$.
Therefore, for any $\lambda \in (-\sqrt{3}, \sqrt{3})$,
there exists a unique $x^*$ such that
$f(x^*, \lambda) = 0$.
Let $x^*(\lambda) : (-\sqrt{3}, \sqrt{3}) \to \mathbb{R}$ be this solution.

\begin{itemize}
	\item We have $x^*(1) = 1$, find the value of derivative $\partial x^*(1)$.
	
	\begin{solutionbox}{7cm}
        We have
        \[
            \partial x^*(\lambda)
            = \frac{-\partial_2 f(1, 1)}{\partial_1 f(1, 1)}
            = \frac{(x^*)^2}{3(x^*)^2 - 2\lambda x^* + 1}
            = \frac{1}{2}
        \]
	\end{solutionbox}
	\item What is the value of the second derivative $\partial^2 x^*(1)$ ?
	
	\begin{solutionbox}{9.5cm}
        Let $n, d$ be the numerator and denominator:
        \[
            n = (x^*)^2
            \qquad
            d = 3(x^*)^2 - 2\lambda x^* + 1.
        \]
        We have
        \[
            \partial^2 x^*(\lambda)
            = \frac{n'd - nd'}{d^2}
        \]
        where $n' = 2x^*(x^*)' = 2 \cdot 1 \cdot (1/2) = 1$,
        $d = 2$, $n = 1$ and
        \[ d' = 6x^*(x^*)' - 2\lambda' x^* - 2\lambda (x^*)'
        = 3 - 2 - 2/2 = 0 \]
        so
        \[
            \partial^2 x^*(\lambda)
            = \frac{1 \cdot 2 - 1 \cdot 0}{2^2} = \frac{2}{4} = \frac{1}{2}.
        \]
	\end{solutionbox}
\end{itemize}	
